\documentclass{article}

\usepackage{booktabs}
\usepackage{tabularx}

\title{Development Plan\\\progname}

\author{\authname}

\date{}

\input{../Comments}
%% Common Parts

\newcommand{\progname}{Software Engineering} % PUT YOUR PROGRAM NAME HERE
\newcommand{\authname}{Team 11, technically functional
\\ Matthew Huynh
\\ Cieran Diebolt
\\ Vaisnavi Shanthamoorthy 
\\ Maham Siddiqui
\\ Eman Ashraf}% AUTHOR NAMES                  

\usepackage{hyperref}
    \hypersetup{colorlinks=true, linkcolor=blue, citecolor=blue, filecolor=blue,
                urlcolor=blue, unicode=false}
    \urlstyle{same}
                                


\begin{document}

\maketitle

\begin{table}[hp]
\caption{Revision History} \label{TblRevisionHistory}
\begin{tabularx}{\textwidth}{llX}
\toprule
\textbf{Date} & \textbf{Developer(s)} & \textbf{Change}\\
\midrule
Date2 & Name(s) & Description of changes\\
... & ... & ...\\
\bottomrule
\end{tabularx}
\end{table}

\newpage{}


This document contains details pertaining to the proposed development plan by the team.
Sections 1 - 3 contain legal and regulation information, sections 4 - 6 contain team dynamics related information and 
section 7-10 contain timeline and technology logistics. The team charter is also included within this document. \\


\section{Confidential Information?}
There is no confidential information involved in this project. Any information
from supervisors or stakeholders is trade-wide information and contains
no trade-secrets or other types of confidential information.

\section{IP to Protect}
Guaranteed intellectual property for the project consists solely of written
content, namely the code and documentation. This written content is covered
under our copyright agreement; more detail in the copyright license section
below.

No additional creations requiring a trademark agreement or any further
protection of intellectual property, such as logos or slogans, are foreseen
at this point of the project. 



\section{Copyright License}
The copyright license adopted by the team for this project is the MIT license.
This license provides others the right to use, copy, modify, merge, publish,
distribute, sublicense, and/or sell copies of the software under the necessity
that the copyright notice and permissions are provided in all copies of the software.

The \href{https://github.com/M9Huynh/technically-functional/blob/f05da0acf59e6b8a08d67f9132471c1a17b5afef/LICENSE}{copyright license} for the project can be found in the team repository.



\section{Team Meeting Plan}

The team will be meeting in person on a weekly basis in a library at McMaster University. The time of the meetings
needs to be adjusted in order to accommodate all the members' time constraints.
Additional virtual meetings may be required when approaching a deliverable deadline.\\
Industry supervisor meetings will be held virtually on a biweekly or as needed basis. \\

Matthew H. will be in charge of the agenda. The rest of the team will be recording meeting minutes interchangeably. All members will be encouraged
to chair meetings throughout the duration of the project. 

\section{Team Communication Plan}

\wss{Issues on GitHub should be part of your communication plan.}


\section{Team Member Roles}

\textbf{Editor}: This member will read through all deliverables throughout development and a final read-through before submission. They will be responsible for any
text alterations to ensure conciseness and effective communication within documents/code.

\textbf{Liaison}: This member will be responsible for communicating with all external parties and relaying their comments back to the team.

\textbf{Researcher}: They will conduct initial research before the team commences a task and convey their findings to the rest of the team.

\textbf{Task Manager}: They will decide how to allocate the workload between the members.

\textbf{Github Manager}: They will ensure that all proposed rules pertaining to the use of Github are followed and that information provided on the platform is complete and precise.

\textbf{Back-end developer}: They will be developing and ensuring that the backend is functional and up to code standards.

\textbf{Front-end developer}: They will be responsible for development of a user friendly interface.

\textbf{Comment Reviewer}: This member will review code and verify the presence of and clarity of comments. Maham will be reviewing the final version.

\textbf{Github Wiki Manager}: This member is responsible for verifying the appropriateness of all content on the Github Wiki.

\begin{table}[hp]
\caption{Member Roles} \label{Proposed Assignments}
\begin{tabularx}{\textwidth}{llX}
\toprule
\textbf{Role} & \textbf{Name} & \textbf{Possibility of Change}\\
\midrule
Editor & All & Yes\\
Liaison & Matthew H  & No\\
Researcher & All & Yes \\
Task Manager & Eman A & Yes \\
Front end developer & All & Yes \\
Back end developer & All & NA \\
Github Manager & Cieran D & Yes \\
Comment Reviewer & All & NA \\
Github Wiki Manager & All & NA \\
\bottomrule
\end{tabularx}
\end{table}



\section{Workflow Plan}

We will primarily be using GitHub for managing the project files. \\
Each team member has their individual branches on which they will engage with. 
The member should open a pull request after completion of their assigned task. If 
multiple members are collaborating on the same task, they will use either member's branch. Once completed, 
a pull request can be opened which will be reviewed by the other team members. \\

Issues that focus on a single aspect of the project may be opened and will combine
all relevant items. Sub-issues can be used if additional refinement is needed.


\begin{itemize}
	\item How will you be using git, including branches, pull request, etc.?
	\item How will you be managing issues, including template issues, issue
	classification, etc.?
  \item Use of CI/CD
\end{itemize}

\section{Project Decomposition and Scheduling}

\begin{itemize}
  \item How will you be using GitHub projects?
  \item Include a link to your GitHub project
\end{itemize}

\wss{How will the project be scheduled?  This is the big picture schedule, not
details. You will need to reproduce information that is in the course outline
for deadlines.}

\section{Proof of Concept Demonstration Plan}

What is the main risk, or risks, for the success of your project?  What will you
demonstrate during your proof of concept demonstration to convince yourself that
you will be able to overcome this risk?

\section{Expected Technology}

\wss{What programming language or languages do you expect to use?  What external
libraries?  What frameworks?  What technologies.  Are there major components of
the implementation that you expect you will implement, despite the existence of
libraries that provide the required functionality.  For projects with machine
learning, will you use pre-trained models, or be training your own model?  }

\wss{The implementation decisions can, and likely will, change over the course
of the project.  The initial documentation should be written in an abstract way;
it should be agnostic of the implementation choices, unless the implementation
choices are project constraints.  However, recording our initial thoughts on
implementation helps understand the challenge level and feasibility of a
project.  It may also help with early identification of areas where project
members will need to augment their training.}

Topics to discuss include the following:

\begin{itemize}
\item Specific programming language
\item Specific libraries
\item Pre-trained models
\item Specific linter tool (if appropriate)
\item Specific unit testing framework
\item Investigation of code coverage measuring tools
\item Specific plans for Continuous Integration (CI), or an explanation that CI
  is not being done
\item Specific performance measuring tools (like Valgrind), if
  appropriate
\item Tools you will likely be using?
\end{itemize}

\wss{git, GitHub and GitHub projects should be part of your technology.}

\section{Coding Standard}

\wss{What coding standard will you adopt?}

\newpage{}

\section*{Appendix --- Reflection}

\input{../Reflection.tex}

\begin{enumerate}
    \item Why is it important to create a development plan prior to starting the
    project?
    \item In your opinion, what are the advantages and disadvantages of using
    CI/CD?
    \item What disagreements did your group have in this deliverable, if any,
    and how did you resolve them?
\end{enumerate}
\textbf{Maham} \\
Having a plan eliminates a lot of uncertainty, whether it is timeline, team roles or conflict resolution methodologies. It allows the team to focus on what is important without being bogged down by frivolous details. Plus, it allows effective allocation of resources. For example, if one person is better at front end development, members who do not want to be or cannot be involved in it, do not have to start from scratch and master a new skill within a small time frame. 

The main advantage of CI/CD is a smoother integration of code while reducing the amount of mistakes and code conflicts that would need to be fixed at the end of the project. However, there is one troublesome disadvantage - conflicting merge requests (this is in the context of using github) and the very time consuming endeavour of fixing them. 

As for group disagreements, I do not think we had any. All members carefully listened to any input from their teammates and implemented any changes that were deemed to be beneficial. We were professional and respectful of each other’s opinions and gave equal importance to everyone’s feedback. \\


\newpage{}

\section*{Appendix --- Team Charter}

\wss{borrows from
\href{https://engineering.up.edu/industry_partnerships/files/team-charter.pdf}
{University of Portland Team Charter}}

\subsection*{External Goals}

\wss{What are your team's external goals for this project? These are not the
goals related to the functionality or quality fo the project.  These are the
goals on what the team wishes to achieve with the project.  Potential goals are
to win a prize at the Capstone EXPO, or to have something to talk about in
interviews, or to get an A+, etc.}

\subsection*{Attendance}

\subsubsection*{Expectations}

\wss{What are your team's expectations regarding meeting attendance (being on
time, leaving early, missing meetings, etc.)?}

\subsubsection*{Acceptable Excuse}

\wss{What constitutes an acceptable excuse for missing a meeting or a deadline?
What types of excuses will not be considered acceptable?}

\subsubsection*{In Case of Emergency}

\wss{What process will team members follow if they have an emergency and cannot
attend a team meeting or complete their individual work promised for a team
deliverable?}

\subsection*{Accountability and Teamwork}

\subsubsection*{Quality} 

\wss{What are your team's expectations regarding the quality
of team members' preparation for team meetings and the quality of the
deliverables that members bring to the team?}

\subsubsection*{Attitude}

\wss{What are your team's expectations regarding team members' ideas,
interactions with the team, cooperation, attitudes, and anything else regarding
team member contributions?  Do you want to introduce a code of conduct?  Do you
want a conflict resolution plan?  Can adopt existing codes of conduct.}

\subsubsection*{Stay on Track}

\wss{What methods will be used to keep the team on track? How will your team
ensure that members contribute as expected to the team and that the team
performs as expected? How will your team reward members who do well and manage
members whose performance is below expectations?  What are the consequences for
someone not contributing their fair share?}

\wss{You may wish to use the project management metrics collected for the TA and
instructor for this.}

\wss{You can set target metrics for attendance, commits, etc.  What are the
consequences if someone doesn't hit their targets?  Do they need to bring the
coffee to the next team meeting?  Does the team need to make an appointment with
their TA, or the instructor?  Are there incentives for reaching targets early?}

\subsubsection*{Team Building}

The team will engage in social activities together if schedules allow. Team building activities will strongly be encouraged. There may be sessions of working together in person and spending some time together with any member that is available. This will also depend on work habits and preferences.

\subsubsection*{Decision Making} 

\wss{How will you make decisions in your group? Consensus?  Vote? How will you
handle disagreements? }

\end{document}