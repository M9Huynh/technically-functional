\documentclass{article}

\usepackage{booktabs}
\usepackage{tabularx}

\title{Development Plan\\\progname}

\author{\authname}

\date{}

%% Comments

\usepackage{color}

\newif\ifcomments\commentstrue %displays comments
%\newif\ifcomments\commentsfalse %so that comments do not display

\ifcomments
\newcommand{\authornote}[3]{\textcolor{#1}{[#3 ---#2]}}
\newcommand{\todo}[1]{\textcolor{red}{[TODO: #1]}}
\else
\newcommand{\authornote}[3]{}
\newcommand{\todo}[1]{}
\fi

\newcommand{\wss}[1]{\authornote{magenta}{SS}{#1}} 
\newcommand{\plt}[1]{\authornote{cyan}{TPLT}{#1}} %For explanation of the template
\newcommand{\an}[1]{\authornote{cyan}{Author}{#1}}

%% Common Parts

\newcommand{\progname}{ProgName} % PUT YOUR PROGRAM NAME HERE
\newcommand{\authname}{Team \#, Team Name
\\ Student 1 name
\\ Student 2 name
\\ Student 3 name
\\ Student 4 name} % AUTHOR NAMES                  

\usepackage{hyperref}
    \hypersetup{colorlinks=true, linkcolor=blue, citecolor=blue, filecolor=blue,
                urlcolor=blue, unicode=false}
    \urlstyle{same}
                                


\begin{document}

\maketitle

\begin{table}[hp]
\caption{Revision History} \label{TblRevisionHistory}
\begin{tabularx}{\textwidth}{llX}
\toprule
\textbf{Date} & \textbf{Developer(s)} & \textbf{Change}\\
\midrule
September 19th, 2025 & Maham & Preliminary Input \\
Date2 & Name(s) & Description of changes\\
... & ... & ...\\
\bottomrule
\end{tabularx}
\end{table}

\newpage{}

\wss{Put your introductory blurb here.  Often the blurb is a brief roadmap of
what is contained in the report.}
This document contains a proposed development plan. \\

\wss{Additional information on the development plan can be found in the
\href{https://gitlab.cas.mcmaster.ca/courses/capstone/-/blob/main/Lectures/L02b_POCAndDevPlan/POCAndDevPlan.pdf?ref_type=heads}
{lecture slides}.}

\section{Confidential Information?}

\wss{State whether your project has confidential information from industry, or
not.  If there is confidential information, point to the agreement you have in
place.}\\
The application will be recording and analyzing a video of the user performing the exercise. This will 
be addressed in the Terms and Conditions that the user has to accept if they want to use the application. 

\wss{For most teams this section will just state that there is no confidential
information to protect.}
\section{IP to Protect}

\wss{State whether there is IP to protect.  If there is, point to the agreement.
All students who are working on a project that requires an IP agreement are also
required to sign the ``Intellectual Property Guide Acknowledgement.''}

\section{Copyright License}

\wss{What copyright license is your team adopting.  Point to the license in your
repo.}

\section{Team Meeting Plan}

\wss{How often will you meet? where?}
The team will be meeting in person on a weekly basis at the library at McMaster University. The time of the meetings
needs to be adjusted in order to accommodate all the members' time constraints. \\
Additional virtual meetings may be required when approaching a deliverable deadline.

\wss{If the meeting is a physical location (not virtual), out of an abundance of
caution for safety reasons you shouldn't put the location online}

\wss{How often will you meet with your industry advisor?  when?  where?}
The advisor selection process is currently in progress. 

\wss{Will meetings be virtual?  At least some meetings should likely be
in-person.}

\wss{How will the meetings be structured?  There should be a chair for all meetings.  There should be an agenda for all meetings.}
Matthew H. will be in charge of the agenda. The rest of the team will be recording meeting minutes interchangeably. All members will be encouraged
to chair meetings throughout the duration of the project. 

\section{Team Communication Plan}

Our team communication plan will be as follows:
\begin{itemize}
  \item Issues: GitHub/GitHub Projects
  \item Weekly Meetings: In-person on campus
  \item Meetings outside the weekly meetings: Teams
  \item Meeting minutes: GitHub
  \item Meetings with stakeholders: Zoom
  \item Project discussion (asynchronous): Teams
\end{itemize}


\section{Team Member Roles}

\wss{You should identify the types of roles you anticipate, like notetaker,
leader, meeting chair, reviewer.  Assigning specific people to those roles is
not necessary at this stage.  In a student team the role of the individuals will
likely change throughout the year.}

\begin{table}[hp]
\caption{Member Roles} \label{Proposed Assignments}
\begin{tabularx}{\textwidth}{llX}
\toprule
\textbf{Name} & \textbf{Role} & \textbf{Possibility of Change}\\
\midrule
Maham S & Editor & If needed\\
Cieran D & Liaison  & No\\
Vaisnavi S & Researcher & Yes \\
Eman A & Task Manager & Yes \\
Matthew H & Github Manager & Yes \\
\bottomrule
\end{tabularx}
\end{table}


\section{Workflow Plan}

We will primarily be using GitHub for managing the project files. \\
Each team member has their individual branches on which they will engage with. 
The member should open a pull request after completion of their assigned task. If 
multiple members are collaborating on the same task, they will use either member's branch. Once completed, 
a pull request can be opened which will be reviewed by the other team members. \\

Issues that focus on a single aspect of the project may be opened and will combine
all relevant items. Sub-issues can be used if additional refinement is needed.


\begin{itemize}
	\item How will you be using git, including branches, pull request, etc.?
	\item How will you be managing issues, including template issues, issue
	classification, etc.?
  \item Use of CI/CD
\end{itemize}

\section{Project Decomposition and Scheduling}

\begin{itemize}
  \item How will you be using GitHub projects?
  \item Include a link to your GitHub project
\end{itemize}

\wss{How will the project be scheduled?  This is the big picture schedule, not
details. You will need to reproduce information that is in the course outline
for deadlines.}

\section{Proof of Concept Demonstration Plan}

The application will follow the steps below (high-level):
\begin{enumerate}
  \item A user will record them performing a physio exercise (decide on body part?)
  \item Determine if the exercise is “good” form or “bad” form based on requirements from the stakeholders
  \item From the analysis, the application will provide guidance to improve the efficacy of the exercise
\end{enumerate}


Some potential risks from this process flow are:
\begin{itemize}
  \item Variations in how different users will record each video.
  \begin{itemize}
    \item Having a guide overlay to indicate where they should film the exercise from.
  \end{itemize}  
  \item Inaccuracies with depth or monitoring a 3D motion on a 2D plane
  \begin{itemize}
    \item Prompting the user to also perform the exercise from another angle and then analyzing two different recordings.
  \end{itemize}
  \item Hardware limitations - as the application would be easiest to use on a phone due to the camera already being on there, it may have limitations when processing the footage.
  \begin{itemize}
    \item Using a lightweight solution to process the recording to reduce lag.
  \end{itemize}
  \item Accuracy in the footage may be different in a non-controlled environment and lead to inaccurate assessment of the exercise. 
  \begin{itemize}
    \item A demonstration can consist of seeing if we are able to obtain the same feedback in different light levels/clothing to see if it will affect our input data.
  \end{itemize} 
\end{itemize}

\section{Expected Technology}

\wss{What programming language or languages do you expect to use?  What external
libraries?  What frameworks?  What technologies.  Are there major components of
the implementation that you expect you will implement, despite the existence of
libraries that provide the required functionality.  For projects with machine
learning, will you use pre-trained models, or be training your own model?  }

\wss{The implementation decisions can, and likely will, change over the course
of the project.  The initial documentation should be written in an abstract way;
it should be agnostic of the implementation choices, unless the implementation
choices are project constraints.  However, recording our initial thoughts on
implementation helps understand the challenge level and feasibility of a
project.  It may also help with early identification of areas where project
members will need to augment their training.}

Topics to discuss include the following:

\begin{itemize}
\item Specific programming language
\item Specific libraries
\item Pre-trained models
\item Specific linter tool (if appropriate)
\item Specific unit testing framework
\item Investigation of code coverage measuring tools
\item Specific plans for Continuous Integration (CI), or an explanation that CI
  is not being done
\item Specific performance measuring tools (like Valgrind), if
  appropriate
\item Tools you will likely be using?
\end{itemize}

\wss{git, GitHub and GitHub projects should be part of your technology.}

\section{Coding Standard}

\wss{What coding standard will you adopt?}

\newpage{}

\section*{Appendix --- Reflection}

\wss{Not required for CAS 741}

The purpose of reflection questions is to give you a chance to assess your own
learning and that of your group as a whole, and to find ways to improve in the
future. Reflection is an important part of the learning process.  Reflection is
also an essential component of a successful software development process.  

Reflections are most interesting and useful when they're honest, even if the
stories they tell are imperfect. You will be marked based on your depth of
thought and analysis, and not based on the content of the reflections
themselves. Thus, for full marks we encourage you to answer openly and honestly
and to avoid simply writing ``what you think the evaluator wants to hear.''

Please answer the following questions.  Some questions can be answered on the
team level, but where appropriate, each team member should write their own
response:


\begin{enumerate}
    \item Why is it important to create a development plan prior to starting the
    project?
    \item In your opinion, what are the advantages and disadvantages of using
    CI/CD?
    \item What disagreements did your group have in this deliverable, if any,
    and how did you resolve them?
\end{enumerate}

\newpage{}

\section*{Appendix --- Team Charter}

\wss{borrows from
\href{https://engineering.up.edu/industry_partnerships/files/team-charter.pdf}
{University of Portland Team Charter}}

\subsection*{External Goals}

Our group's primary goal is to create a project that we are proud to have our names
on, can be talked about in interviews and to peers. It is important for us to build 
a solution to our problem statement while also adhering to code quality such as 
readability and documentation. Through this, we seek to develop our technical skills 
through tackling a real world problem. Additionally, we are also aiming for above-average 
grades as a secondary goal. 

\subsection*{Attendance}

\subsubsection*{Expectations}

Our team expects full attendance to meetings scheduled in advance for the entire duration 
of the meeting. If a member cannot attend a meeting or is going to miss a portion of the meeting, 
they must communicate it in advance to the team and catch-up on the missed portion of the meeting 
(i.e. meeting minutes). Missing 3 meetings in a row when a deliverable is not the main point of 
discussion will trigger a meeting with the rest of the team. Moreover, if a deliverable is the 
main topic, 2 missed meetings will be the limit. These standards are for non-emergency situations. 
For emergencies, see below. 

\subsubsection*{Acceptable Excuse}

Acceptable excuses include: Academic sessions (classes, tutorials), academic requirements (due-dates
prior to the capstone deliverable) and medical appointments. An advance notice for medical 
appointments will be appreciated. It is the responsibility of the missing member to make up for a 
missed meeting. Repeated excuses will not be acceptable; please refer to attendance expectations. 
Please see below for emergency cases.

\subsubsection*{In Case of Emergency}

In the event of an emergency case, the team member should contact the team. Extended cases of 
emergency (more than 2 weeks or time taken off that impedes deliverables) should be brought to 
the attention of the teaching staff to ensure the work load is manageable for the rest of the team.

\subsection*{Accountability and Teamwork}

\subsubsection*{Quality} 

\wss{What are your team's expectations regarding the quality
of team members' preparation for team meetings and the quality of the
deliverables that members bring to the team?}

\subsubsection*{Attitude}

\wss{What are your team's expectations regarding team members' ideas,
interactions with the team, cooperation, attitudes, and anything else regarding
team member contributions?  Do you want to introduce a code of conduct?  Do you
want a conflict resolution plan?  Can adopt existing codes of conduct.}

\subsubsection*{Stay on Track}

\wss{What methods will be used to keep the team on track? How will your team
ensure that members contribute as expected to the team and that the team
performs as expected? How will your team reward members who do well and manage
members whose performance is below expectations?  What are the consequences for
someone not contributing their fair share?}

\wss{You may wish to use the project management metrics collected for the TA and
instructor for this.}

\wss{You can set target metrics for attendance, commits, etc.  What are the
consequences if someone doesn't hit their targets?  Do they need to bring the
coffee to the next team meeting?  Does the team need to make an appointment with
their TA, or the instructor?  Are there incentives for reaching targets early?}

\subsubsection*{Team Building}

\wss{How will you build team cohesion (fun time, group rituals, etc.)? }

\subsubsection*{Decision Making} 

\wss{How will you make decisions in your group? Consensus?  Vote? How will you
handle disagreements? }

\end{document}