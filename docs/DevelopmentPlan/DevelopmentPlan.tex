\documentclass{article}

\usepackage{booktabs}
\usepackage{tabularx}

\title{Development Plan\\\progname}

\author{\authname}

\date{}

%% Comments

\usepackage{color}

\newif\ifcomments\commentstrue %displays comments
%\newif\ifcomments\commentsfalse %so that comments do not display

\ifcomments
\newcommand{\authornote}[3]{\textcolor{#1}{[#3 ---#2]}}
\newcommand{\todo}[1]{\textcolor{red}{[TODO: #1]}}
\else
\newcommand{\authornote}[3]{}
\newcommand{\todo}[1]{}
\fi

\newcommand{\wss}[1]{\authornote{magenta}{SS}{#1}} 
\newcommand{\plt}[1]{\authornote{cyan}{TPLT}{#1}} %For explanation of the template
\newcommand{\an}[1]{\authornote{cyan}{Author}{#1}}

%% Common Parts

\newcommand{\progname}{ProgName} % PUT YOUR PROGRAM NAME HERE
\newcommand{\authname}{Team \#, Team Name
\\ Student 1 name
\\ Student 2 name
\\ Student 3 name
\\ Student 4 name} % AUTHOR NAMES                  

\usepackage{hyperref}
    \hypersetup{colorlinks=true, linkcolor=blue, citecolor=blue, filecolor=blue,
                urlcolor=blue, unicode=false}
    \urlstyle{same}
                                


\begin{document}

\maketitle

\begin{table}[hp]
\caption{Revision History} \label{TblRevisionHistory}
\begin{tabularx}{\textwidth}{llX}
\toprule
\textbf{Date} & \textbf{Developer(s)} & \textbf{Change}\\
\midrule
Date2 & Name(s) & Description of changes\\
... & ... & ...\\
\bottomrule
\end{tabularx}
\end{table}

\newpage{}


This document contains details pertaining to the proposed development plan by the team.
Sections 1 - 3 contain legal and regulation information, sections 4 - 6 contain team dynamics related information and 
section 7-10 contain timeline and technology logistics. The team charter is also included within this document. \\


\section{Confidential Information?}

\section{IP to Protect}


\section{Copyright License}



\section{Team Meeting Plan}

The team will be meeting in person on a weekly basis in a library at McMaster University. The time of the meetings
needs to be adjusted in order to accommodate all the members' time constraints.
Additional virtual meetings may be required when approaching a deliverable deadline.\\
Industry supervisor meetings will be held virtually on a biweekly or as needed basis. \\

Matthew H. will be in charge of the agenda. The rest of the team will be recording meeting minutes interchangeably. All members will be encouraged
to chair meetings throughout the duration of the project. 

\section{Team Communication Plan}

Our team communication plan will be as follows:
\begin{itemize}
  \item Issues: GitHub/GitHub Projects
  \item Weekly Meetings: In-person on campus
  \item Meetings outside the weekly meetings: Teams
  \item Meeting minutes: GitHub
  \item Meetings with stakeholders: Zoom
  \item Project discussion (asynchronous): Teams
\end{itemize}


\section{Team Member Roles}

\textbf{Editor}: This member will read through all deliverables throughout development and a final read-through before submission. They will be responsible for any
text alterations to ensure conciseness and effective communication within documents/code.

\textbf{Liaison}: This member will be responsible for communicating with all external parties and relaying their comments back to the team.

\textbf{Researcher}: They will conduct initial research before the team commences a task and convey their findings to the rest of the team.

\textbf{Task Manager}: They will decide how to allocate the workload between the members.

\textbf{Github Manager}: They will ensure that all proposed rules pertaining to the use of Github are followed and that information provided on the platform is complete and precise.

\textbf{Back-end developer}: They will be developing and ensuring that the backend is functional and up to code standards.

\textbf{Front-end developer}: They will be responsible for development of a user friendly interface.

\textbf{Comment Reviewer}: This member will review code and verify the presence of and clarity of comments. Maham will be reviewing the final version.

\textbf{Github Wiki Manager}: This member is responsible for verifying the appropriateness of all content on the Github Wiki.

\begin{table}[hp]
\caption{Member Roles} \label{Proposed Assignments}
\begin{tabularx}{\textwidth}{llX}
\toprule
\textbf{Role} & \textbf{Name} & \textbf{Possibility of Change}\\
\midrule
Editor & All & Yes\\
Liaison & Matthew H  & No\\
Researcher & All & Yes \\
Task Manager & Eman A & Yes \\
Front end developer & All & Yes \\
Back end developer & All & NA \\
Github Manager & Cieran D & Yes \\
Comment Reviewer & All & NA \\
Github Wiki Manager & All & NA \\
\bottomrule
\end{tabularx}
\end{table}



\section{Workflow Plan}

We will primarily be using GitHub for managing the project files. \\
Each team member has their individual branches on which they will engage with. 
The member should open a pull request after completion of their assigned task. If 
multiple members are collaborating on the same task, they will use either member's branch. Once completed, 
a pull request can be opened which will be reviewed by the other team members. \\

Issues that focus on a single aspect of the project may be opened and will combine
all relevant items. Sub-issues can be used if additional refinement is needed.


\begin{itemize}
	\item How will you be using git, including branches, pull request, etc.?
	\item How will you be managing issues, including template issues, issue
	classification, etc.?
  \item Use of CI/CD
\end{itemize}

\section{Project Decomposition and Scheduling}

\begin{itemize}
  \item How will you be using GitHub projects?
  \item Include a link to your GitHub project
\end{itemize}

\wss{How will the project be scheduled?  This is the big picture schedule, not
details. You will need to reproduce information that is in the course outline
for deadlines.}

\section{Proof of Concept Demonstration Plan}

The application will follow the steps below (high-level):
\begin{enumerate}
  \item A user will record them performing a physio exercise (decide on body part?)
  \item Determine if the exercise is “good” form or “bad” form based on requirements from the stakeholders
  \item From the analysis, the application will provide guidance to improve the efficacy of the exercise
\end{enumerate}


Some potential risks from this process flow are:
\begin{itemize}
  \item Variations in how different users will record each video.
  \begin{itemize}
    \item Having a guide overlay to indicate where they should film the exercise from.
  \end{itemize}  
  \item Inaccuracies with depth or monitoring a 3D motion on a 2D plane
  \begin{itemize}
    \item Prompting the user to also perform the exercise from another angle and then analyzing two different recordings.
  \end{itemize}
  \item Hardware limitations - as the application would be easiest to use on a phone due to the camera already being on there, it may have limitations when processing the footage.
  \begin{itemize}
    \item Using a lightweight solution to process the recording to reduce lag.
  \end{itemize}
  \item Accuracy in the footage may be different in a non-controlled environment and lead to inaccurate assessment of the exercise. 
  \begin{itemize}
    \item A demonstration can consist of seeing if we are able to obtain the same feedback in different light levels/clothing to see if it will affect our input data.
  \end{itemize} 
\end{itemize}

\section{Expected Technology}

\wss{What programming language or languages do you expect to use?  What external
libraries?  What frameworks?  What technologies.  Are there major components of
the implementation that you expect you will implement, despite the existence of
libraries that provide the required functionality.  For projects with machine
learning, will you use pre-trained models, or be training your own model?  }

\wss{The implementation decisions can, and likely will, change over the course
of the project.  The initial documentation should be written in an abstract way;
it should be agnostic of the implementation choices, unless the implementation
choices are project constraints.  However, recording our initial thoughts on
implementation helps understand the challenge level and feasibility of a
project.  It may also help with early identification of areas where project
members will need to augment their training.}

Topics to discuss include the following:

\begin{itemize}
\item Specific programming language
\item Specific libraries
\item Pre-trained models
\item Specific linter tool (if appropriate)
\item Specific unit testing framework
\item Investigation of code coverage measuring tools
\item Specific plans for Continuous Integration (CI), or an explanation that CI
  is not being done
\item Specific performance measuring tools (like Valgrind), if
  appropriate
\item Tools you will likely be using?
\end{itemize}

\wss{git, GitHub and GitHub projects should be part of your technology.}

\section{Coding Standard}

\wss{What coding standard will you adopt?}

\newpage{}

\section*{Appendix --- Reflection}

The purpose of reflection questions is to give you a chance to assess your own
learning and that of your group as a whole, and to find ways to improve in the
future. Reflection is an important part of the learning process.  Reflection is
also an essential component of a successful software development process.  

Reflections are most interesting and useful when they're honest, even if the
stories they tell are imperfect. You will be marked based on your depth of
thought and analysis, and not based on the content of the reflections
themselves. Thus, for full marks we encourage you to answer openly and honestly
and to avoid simply writing ``what you think the evaluator wants to hear.''

Please answer the following questions.  Some questions can be answered on the
team level, but where appropriate, each team member should write their own
response:


\begin{enumerate}
    \item Why is it important to create a development plan prior to starting the
    project?
    \item In your opinion, what are the advantages and disadvantages of using
    CI/CD?
    \item What disagreements did your group have in this deliverable, if any,
    and how did you resolve them?
\end{enumerate}
\textbf{Maham} \\
Having a plan eliminates a lot of uncertainty, whether it is timeline, team roles or conflict resolution methodologies. It allows the team to focus on what is important without being bogged down by frivolous details. Plus, it allows effective allocation of resources. For example, if one person is better at front end development, members who do not want to be or cannot be involved in it, do not have to start from scratch and master a new skill within a small time frame. 

The main advantage of CI/CD is a smoother integration of code while reducing the amount of mistakes and code conflicts that would need to be fixed at the end of the project. However, there is one troublesome disadvantage - conflicting merge requests (this is in the context of using github) and the very time consuming endeavour of fixing them. 

As for group disagreements, I do not think we had any. All members carefully listened to any input from their teammates and implemented any changes that were deemed to be beneficial. We were professional and respectful of each other’s opinions and gave equal importance to everyone’s feedback. \\


\textbf{Matthew}\\
I think it is important to create a development plan prior to starting the project because it helps our team navigate conflict down the line. Not everyone works the same and so this development plan acts as a formal document to communicate with the team and find common ground for everyone to perform well on. Furthermore, it is a form of having an awkward conversation now compared to down the line when this work wasn't done and there is no basis to critique a member's performance. 

An advantage of CI/CD is the reduced manual effort to test and evaluate code. If there is a problem with one of our documents that we upload or a file that is incorrect, we can have the tests evaluate it and be flagged for issues with it. A downside of CI/CD is the overhead or the amount of time that it takes to set up. For a shorter project it will take a bit longer to set up and thus less time is put towards coding the features at the cost of automating the building and testing.

The group did not have any disagreements during this deliverable. Everyone was able to communicate to each other and each meeting went with a plan and left with actionable items to complete. As the term gets busier and there are more deliverables for other classes I think that the team charter below will help guide us for the rest of the project.



\newpage{}

\section*{Appendix --- Team Charter}

\wss{borrows from
\href{https://engineering.up.edu/industry_partnerships/files/team-charter.pdf}
{University of Portland Team Charter}}

\subsection*{External Goals}

Our group's primary goal is to create a project that we are proud to have our names
on, can be talked about in interviews and to peers. It is important for us to build 
a solution to our problem statement while also adhering to code quality such as 
readability and documentation. Through this, we seek to develop our technical skills 
through tackling a real world problem. Additionally, we are also aiming for above-average 
grades as a secondary goal. 

\subsection*{Attendance}

\subsubsection*{Expectations}

Our team expects full attendance to meetings scheduled in advance for the entire duration 
of the meeting. If a member cannot attend a meeting or is going to miss a portion of the meeting, 
they must communicate it in advance to the team and catch-up on the missed portion of the meeting 
(i.e. meeting minutes). Missing 3 meetings in a row when a deliverable is not the main point of 
discussion will trigger a meeting with the rest of the team. Moreover, if a deliverable is the 
main topic, 2 missed meetings will be the limit. These standards are for non-emergency situations. 
For emergencies, see below. 

\subsubsection*{Acceptable Excuse}

Acceptable excuses include: Academic sessions (classes, tutorials), academic requirements (due-dates
prior to the capstone deliverable) and medical appointments. An advance notice for medical 
appointments will be appreciated. It is the responsibility of the missing member to make up for a 
missed meeting. Repeated excuses will not be acceptable; please refer to attendance expectations. 
Please see below for emergency cases.

\subsubsection*{In Case of Emergency}

In the event of an emergency case, the team member should contact the team. Extended cases of 
emergency (more than 2 weeks or time taken off that impedes deliverables) should be brought to 
the attention of the teaching staff to ensure the work load is manageable for the rest of the team.

\subsection*{Accountability and Teamwork}

\subsubsection*{Quality} 

In order to maintain consistent quality for deliverables, each weekly meeting will briefly go over 
deliverable expectations. Each deliverable should be checked against the rubric [on avenue] and also 
against the repository checklist by the individual that wrote that deliverable but also by a teammate. 


Furthermore, for meetings, team members are expected to arrive, review relevant material and provide 
any problems or status updates with their work to ensure the meeting stays on-topic and productive.

Lastly, during weekly meetings we will discuss “what went well, what to change” to see if there is a 
way to streamline communication to attain this quality. We also encourage each team member to ask for 
help in our informal chat in a timely manner while being respectful of each person's time.


\subsubsection*{Attitude}

\wss{What are your team's expectations regarding team members' ideas,
interactions with the team, cooperation, attitudes, and anything else regarding
team member contributions?  Do you want to introduce a code of conduct?  Do you
want a conflict resolution plan?  Can adopt existing codes of conduct.}

\subsubsection*{Stay on Track}

\wss{What methods will be used to keep the team on track? How will your team
ensure that members contribute as expected to the team and that the team
performs as expected? How will your team reward members who do well and manage
members whose performance is below expectations?  What are the consequences for
someone not contributing their fair share?}

\wss{You may wish to use the project management metrics collected for the TA and
instructor for this.}

\wss{You can set target metrics for attendance, commits, etc.  What are the
consequences if someone doesn't hit their targets?  Do they need to bring the
coffee to the next team meeting?  Does the team need to make an appointment with
their TA, or the instructor?  Are there incentives for reaching targets early?}

\subsubsection*{Team Building}

The team will engage in social activities together if schedules allow. Team building activities will strongly be encouraged. There may be sessions of working together in person and spending some time together with any member that is available. This will also depend on work habits and preferences.

\subsubsection*{Decision Making} 

When making a decision on the team, we will lean towards a unanimous vote. We 
want each member of the team to feel their feedback is heard to improve the project 
communication. In cases where this is not the case, we will hold a vote. Since there 
are 5 members, no one can abstain that way a solution can be achieved. 


Each member is allowed to voice their opinion to the team without interruption so 
everyone is heard and we can reach an objective solution (without personal feelings 
swaying the decision). 


In the event that voting cannot be held properly, we will select a neutral mediator to 
guide us further.


\end{document}