\documentclass{article}
\usepackage{natbib}
\usepackage{tabularx}
\usepackage{booktabs}

\title{Problem Statement and Goals\\\progname}

\author{\authname}

\date{}

%% Comments

\usepackage{color}

\newif\ifcomments\commentstrue %displays comments
%\newif\ifcomments\commentsfalse %so that comments do not display

\ifcomments
\newcommand{\authornote}[3]{\textcolor{#1}{[#3 ---#2]}}
\newcommand{\todo}[1]{\textcolor{red}{[TODO: #1]}}
\else
\newcommand{\authornote}[3]{}
\newcommand{\todo}[1]{}
\fi

\newcommand{\wss}[1]{\authornote{magenta}{SS}{#1}} 
\newcommand{\plt}[1]{\authornote{cyan}{TPLT}{#1}} %For explanation of the template
\newcommand{\an}[1]{\authornote{cyan}{Author}{#1}}

%% Common Parts

\newcommand{\progname}{ProgName} % PUT YOUR PROGRAM NAME HERE
\newcommand{\authname}{Team \#, Team Name
\\ Student 1 name
\\ Student 2 name
\\ Student 3 name
\\ Student 4 name} % AUTHOR NAMES                  

\usepackage{hyperref}
    \hypersetup{colorlinks=true, linkcolor=blue, citecolor=blue, filecolor=blue,
                urlcolor=blue, unicode=false}
    \urlstyle{same}
                                


\begin{document}

\maketitle

\begin{table}[hp]
\caption{Revision History} \label{TblRevisionHistory}
\begin{tabularx}{\textwidth}{llX}
\toprule
\textbf{Date} & \textbf{Developer(s)} & \textbf{Change}\\
\midrule
September 16th, 2025 & Maham & Added preliminary stakeholder information\\
September 17th, 2025 & Matthew & Added 1.1, 1.2 and References\\
September 17th, 2025 & Vaisnavi & Added onto 1.2 and Reflection\\
... & ... & ...\\
\bottomrule
\end{tabularx}
\end{table}

\section{Problem Statement}

\subsection{Problem}

According to the Global Burden of Diseases, Injuries and Risk Factors study performed in 2019, 
individuals that would benefit from physical rehabilitation at least once in their 
lifetime is upwards of 2.41 billion globally \citep{CiezaEtAl2021}.
Those with access to a physiotherapist experienced a 
disconnect with performing a required movement with proper time-under-tension (TUT) and 
correct form \citep{FaberEtAl2015}. While a physiotherapist can advise these individuals
during their assessments and proceeding follow-up appointments, the efficacy of rehabilitation
depends heavily on the individual's correct performance of the exercise. In turn, this creates a need for a tool that can ensure users 
correctly perform the exercise without supervision. This project aims to develop a tool that can 
provide feedback and corrections for the prescribed physical rehabilitation exercise.

\subsection{Inputs and Outputs}

\textbf{Inputs:} A recording of the user performing their physical rehabilitation exercise, captured through a smartphone, webcam, or any other device. \\
\textbf{Outputs:} Feedback or corrections of the demonstrated movement, along with highlighting targeted adjustments to the form as needed. 

\subsection{Stakeholders}
  \subsubsection{Primary Stakeholders}
    \textbf{End users/Patients:} \\
      The main audience for this application will be users who have been undergoing physiotherapy treatment for their right leg, and have been given a home exercise plan which outlines \textbf(exercise 1). \\    
     
  \subsubsection{Secondary Stakeholders}
    \textbf{Physiotherapists:} \\
      The application can be used as an adjunct tool for physiotherapists, allowing them to evaluate the patient's performance and provide appropriate feedback. 
      
  \subsubsection{Tertiary Stakeholders}
    \textbf{Regulatory authorities:}
      They will be ensuring and assessing that the application is working properly and providing accurate feedback. \\
     \textbf{Other healthcare providers:}
       Specialists such as physiatrists and registered massage therapists may benefit from the information provided by the application about their patients. 

\subsection{Environment}
  \subsubsection{Software}
    The application will be built using a Python environment, with the use of OpenCV and MediaPipe. \\
  \subsubsection{Hardware}
    The application will run on an iOS device with a camera recording ability.

\section{Goals}

\section{Stretch Goals}

\section{Extras}

\wss{For CAS 741: State whether the project is a research project. This
designation, with the approval (or request) of the instructor, can be modified
over the course of the term.}

\wss{For SE Capstone: List your extras.  Potential extras include usability
testing, code walkthroughs, user documentation, formal proof, GenderMag
personas, Design Thinking, etc.  (The full list is on the course outline and in
Lecture 02.) Normally the number of extras will be two.  Approval of the extras
will be part of the discussion with the instructor for approving the project.
The extras, with the approval (or request) of the instructor, can be modified
over the course of the term.}

\newpage{}

\section*{Appendix --- Reflection}



The purpose of reflection questions is to give you a chance to assess your own
learning and that of your group as a whole, and to find ways to improve in the
future. Reflection is an important part of the learning process.  Reflection is
also an essential component of a successful software development process.  

Reflections are most interesting and useful when they're honest, even if the
stories they tell are imperfect. You will be marked based on your depth of
thought and analysis, and not based on the content of the reflections
themselves. Thus, for full marks we encourage you to answer openly and honestly
and to avoid simply writing ``what you think the evaluator wants to hear.''

Please answer the following questions.  Some questions can be answered on the
team level, but where appropriate, each team member should write their own
response:


\begin{enumerate}
    \item What went well while writing this deliverable? 

    We were able to define the problem clearly by looking at research on 'access to physiotherapy' unsupervised performance of physio therapy exercises. Having 
    two members who were more knowledgeable 
    about the topic helped guide our discussions while also piquing our interests as well. At the same time, the 
    rest of the team 'put in a fair amount of research' also researched the topic, which 'allowed us to fully understand' aided in understanding the problem and align 
    on realistic ''scope and'' goals. Moreover, splitting up each of the deliverable sections also made it easier to combine 
    different perspectives.

    \item What pain points did you experience during this deliverable, and how
    did you resolve them?

    One of the major pain points was that 'our original scope was too broad, as we were trying to cover the entire body' the entire body was too broad of a scope. To 
    make the project more manageable, 'we decided to narrow our focus to a specific area of the body and concentrate on
    measurable aspects such as time-under-tension and form correction.  the scope was narrowed to a specific body part where measurements such as time-under-tension

    \item How did you and your team adjust the scope of your goals to ensure
    they are suitable for a Capstone project (not overly ambitious but also of
    appropriate complexity for a senior design project)?

    To ensure our scope of goals remained realistic within the bounds of our timeline, we decided to focus on a specific area 
    of the body rather than the entire thing. We also decided to stick to a small set of exercises to ensure that the project
    can be completed within our timeline.
\end{enumerate}  

\bibliographystyle {plainnat}
\bibliography{../../refs/References}

\end{document}