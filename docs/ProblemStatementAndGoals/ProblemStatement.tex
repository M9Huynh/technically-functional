\documentclass{article}
\usepackage{natbib}
\usepackage{tabularx}
\usepackage{booktabs}

\title{Problem Statement and Goals\\\progname}

\author{\authname}

\date{}

\input{../Comments}
%% Common Parts

\newcommand{\progname}{Software Engineering} % PUT YOUR PROGRAM NAME HERE
\newcommand{\authname}{Team 11, technically functional
\\ Matthew Huynh
\\ Cieran Diebolt
\\ Vaisnavi Shanthamoorthy 
\\ Maham Siddiqui
\\ Eman Ashraf}% AUTHOR NAMES                  

\usepackage{hyperref}
    \hypersetup{colorlinks=true, linkcolor=blue, citecolor=blue, filecolor=blue,
                urlcolor=blue, unicode=false}
    \urlstyle{same}
                                


\begin{document}

\maketitle

\begin{table}[hp]
\caption{Revision History} \label{TblRevisionHistory}
\begin{tabularx}{\textwidth}{llX}
\toprule
\textbf{Date} & \textbf{Developer(s)} & \textbf{Change}\\
\midrule
September 16th, 2025 & Maham & Added preliminary stakeholder information\\
September 17th, 2025 & Matthew & Added 1.1, 1.2 and References\\
September 17th, 2025 & Vaisnavi & Added onto 1.2 and Reflection\\
September 18th, 2025 & All members & TA meeting and feedback\\
September 19th, 2025 & Maham & Edits \\
September 20th, 2025 & Maham & Completing assigned parts and edits. \\
%September 22nd, 2025 & All & Finished First Draft \\
... & ... & ...\\
\bottomrule
\end{tabularx}
\end{table}

\section{Problem Statement}

\subsection{Problem}

According to the Global Burden of Diseases, Injuries and Risk Factors study performed in 2019, 
individuals that would benefit from physical rehabilitation at least once in their 
lifetime is upwards of 2.41 billion globally \citep{CiezaEtAl2021}.
Those with access to a physiotherapist experienced a 
disconnect with performing a required movement with proper time-under-tension (TUT) and 
correct form \citep{FaberEtAl2015}. While a physiotherapist can advise these individuals
during their assessments and proceeding follow-up appointments, the efficacy of rehabilitation
depends heavily on the individual's correct performance of the exercise. In turn, this creates a need for a tool that can ensure users 
correctly perform the exercise without supervision. This project aims to develop a tool that can 
provide feedback and corrections for the prescribed physical rehabilitation exercise.

\subsection{Inputs and Outputs}

\textbf{Inputs:} A recording of the user performing their physical rehabilitation exercise, captured through a smartphone, webcam, or any other device. \\
\textbf{Outputs:} Feedback or corrections of the demonstrated movement, along with highlighting targeted adjustments to the form as needed. 

\subsection{Stakeholders}
  \subsubsection{Primary Stakeholders}
    \textbf{End users/Patients:} \\
      The main audience for this application will be users who have been undergoing physiotherapy treatment for their right leg, and have been given a home exercise plan which outlines the exercise selected by the team. 
      These users will be obtaining accurate feedback and corrections to assist their performance of the exercise from the comfort of their home and without constant input from a physiotherapist.
     
  \subsubsection{Secondary Stakeholders}
    \textbf{Physiotherapists:} \\
      The application can be used as an adjunct tool for physiotherapists, allowing them to evaluate patient performance, recovery changes and whether any modification to the exercise is required. \\
      
  \subsubsection{Tertiary Stakeholders}
    \textbf{Regulatory authorities:}
      They will be ensuring and assessing that the application is working in an ethical manner, safeguarding any patient information that is used and ensuring that accurate results and expertise is being provided. \\
    \textbf{Other healthcare providers:}
       Specialists such as physiatrists and registered massage therapists may benefit from the information provided by the application about their patients. 

\subsection{Environment}
  \subsubsection{Software}
    The application will be built using an object oriented programming language with supplementary libraries to enable the use of computer vision technology. \\
  \subsubsection{Hardware}
    The application will run on an Android device with a camera recording ability.

\section{Goals}

\section{Stretch Goals}

\section{Extras}

The extras chosen for the project are a \textbf{Design Thinking Document}
and \textbf{User Instructional Video}.

The Design Thinking Document will explain how we approached our project
from a higher level perspective.

The User Instructional Video will be a demonstration on the use of our
application, to serve as a guide to new users and a demonstration to
instructors.

\newpage{}

\section*{Appendix --- Reflection}



\input{../Reflection.tex}
\subsection*{Vaisnavi - Reflection}
\begin{enumerate}
    \item What went well while writing this deliverable? 

    The problem was defined clearly by looking at research on unsupervised performance of physiotherapy exercises. Having two members in the team that were more knowledgeable 
    about the topic helped guide discussions and becoming intrigued by the proposed project. Once the team was in agreement of which direction the project was heading, additional research on the topic was done, which aided in understanding the problem and align 
    on a realistic scope and goals. 
    To allocate the required tasks fairly, the deliverable was sub-divided into manageable sections amongst the members. This strategy provided an additional benefit, and allowed seamless integration of various perspectives as a result. 

    \item What pain points did you experience during this deliverable, and how
    did you resolve them?

    One of the major pain points was that the entire body was too broad of a scope. To make the project more manageable, the scope was narrowed to a specific body part, where measurements such as time-under-tension (TUT) and form. TUT will be measured in seconds, and form will be measured using angles and range of movement.

    \item How did you and your team adjust the scope of your goals to ensure
    they are suitable for a Capstone project (not overly ambitious but also of
    appropriate complexity for a senior design project)?
    
    As mentioned above, the focus was shifted to a singular body part as opposed the full body. This project involves the integration 
    of multiple technologies related to computer vision into a functional application, which is a complex endeavour. 
   The goals listed above will become more specific and concrete as the project development progresses. \\
   
   \textbf{Maham}: This project is a challenging endeavour and requires the use and integration of various systems that we have learned over the course of our degree. At first, our project seemed too ambitious when the whole body was being considered, but narrowing it down
   made it more achievable. Some body parts, such as the hand or ankle, were too complex and had too many motions to account for. In the end, it was collaboratively decided that only one exercise would be chosen and that our project would be a 'blueprint' for future development of similar applications. 
   
\end{enumerate}  


\subsection*{Matthew - Reflection}
\begin{enumerate}
    \item What went well while writing this deliverable? 

    I think during this deliverable we were able select a problem statement that resonated with each of us.
    Furthermore, the supporting sections helped guide our conversations and team meetings involving stakeholders,
    environment, inputs and outputs. Something that worked well involved splitting up the sections and subsections
    by topic rather than the deliverable itself. This resulted in more meaningful conversation and each of us acted
    almost as a subject matter expert. This was showcased during the meeting with the TA where we were able to ask
    our own questions while also letting other team members jump in to expand the idea further.

    \item What pain points did you experience during this deliverable, and how
    did you resolve them?

    One main pain point was refining the scope of the project and also project selection which had led to the work
    of this deliverable. Initially starting we had a few ideas around accessibility and "- a tool to assist [blank]" which
    had been good during the brainstorming process as we could begin to filter out projects through a voting process.
    But as the project came down to two, we were divide and navigating it in terms of complexity was difficult. We resolved
    it by building out the scopes of each project in separate 'mini' teams and then presented it informally to each other.
    Through addressing each of our concerns, we were also able to refine the scope so the projects were less prone to
    scope creep later on. Our final decision came down to a vote and we had then resolved our pain point regarding project
    selection and scope.

    \item How did you and your team adjust the scope of your goals to ensure
    they are suitable for a Capstone project (not overly ambitious but also of
    appropriate complexity for a senior design project)?

    I think that in order to adjust the scope of the project, we needed to outline the metrics that we were evaluating for 
    success. Further, we needed to first identify the problem at hand. Once we were able to restrict the problem statement,
    we considered the target stakeholders on both sides [for physiotherapy]. We then considered possible solutions and then
    tried to evaluate the scope. We had to limit the scope quite a bit from what was initially thought but in doing so, we
    created a "MVP" with additional features if there is extra time to develop. For example, our app may be limited to a 
    certain body part for physio assessment but if there is more time then we may expand on the goals and evaluate the 
    stretch goals as well.
\end{enumerate}  

\subsection*{Cieran - Reflection}
\begin{enumerate}
    \item What went well while writing this deliverable? 

    The team came together very early on to divide up the work and set expectations for each other. This set a tone of
    professionality and efficiency that I am looking forward to seeing from the group as we move forward with the project.
    Additionally, everyone was settling in very nicely and this acted as a good re-introduction for a few of us to github's
    functionality and little quirks that we will have to work with and around this term. I think we've done well to manage
    our time and efforts towards this deliverable and I hope it continues.

    \item What pain points did you experience during this deliverable, and how
    did you resolve them?

    Initially we were torn between two projects of different scope and content entirely, unsure of how to proceed we
    stalled for a long time waiting for information from third parties before innevitably taking a vote and crossing the
    issue off ourselves. This wasn't a huge issue, but I would say before we managed to understand and contribute well together
    using github it slowed us down with our parallel work. We came together to find an alternative path to parallel contributions
    and feel as though our progress is well on its way.

    \item How did you and your team adjust the scope of your goals to ensure
    they are suitable for a Capstone project (not overly ambitious but also of
    appropriate complexity for a senior design project)?
    
    We estimated through looking at the pre-approved projects and discussing with other groups that a project using cameras to
    record and measure exercises effectively would be challenging enough and complex enough for our capstone. However, once we
    sat and discussed the problem we determined that the number of exercises and movements we will have to detect and recognize
    is much too large. After that conversation, we locked our scope to a single body part for the time being to prove that this
    project can be done, but also to make the workload reasonable for the team.
    
\end{enumerate}  

\bibliographystyle {plainnat}
\bibliography{../../refs/References}

\end{document}