\documentclass{article}
\usepackage{natbib}
\usepackage{tabularx}
\usepackage{booktabs}

\title{Problem Statement and Goals\\\progname}

\author{\authname}

\date{}

\input{../Comments}
%% Common Parts

\newcommand{\progname}{Software Engineering} % PUT YOUR PROGRAM NAME HERE
\newcommand{\authname}{Team 11, technically functional
\\ Matthew Huynh
\\ Cieran Diebolt
\\ Vaisnavi Shanthamoorthy 
\\ Maham Siddiqui
\\ Eman Ashraf}% AUTHOR NAMES                  

\usepackage{hyperref}
    \hypersetup{colorlinks=true, linkcolor=blue, citecolor=blue, filecolor=blue,
                urlcolor=blue, unicode=false}
    \urlstyle{same}
                                


\begin{document}

\maketitle

\begin{table}[hp]
\caption{Revision History} \label{TblRevisionHistory}
\begin{tabularx}{\textwidth}{llX}
\toprule
\textbf{Date} & \textbf{Developer(s)} & \textbf{Change}\\
\midrule
September 17th, 2025 & Matthew & Added 1.1, 1.2 and References\\
Date2 & Name(s) & Description of changes\\
... & ... & ...\\
\bottomrule
\end{tabularx}
\end{table}

\section{Problem Statement}

\subsection{Problem}

According to the Global Burden of Diseases, Injuries and Risk Factors study performed in 2019, 
individuals that would benefit from physical rehabilitation at least once in their 
lifetime is upwards of 2.41 billion globally \citep{CiezaEtAl2021}.
Furthermore, a study indicating the perception of access to physiotherapy, based on 
socio-demographic factors, yielded that approximately 1 in 4 participants 
felt they had limited access to physiotherapy as a result of cost, wait-times or 
location \citep{BathEtAl2016}. Those with access to a physiotherapist experienced a 
disconnect with performing a required movement with proper time-under-tension (TUT) and 
correct form \citep{FaberEtAl2015}. While a physiotherapist can advise these individuals
during their assessments and proceeding follow-up appointments, the efficacy of rehabilitation
depends heavily on the individual's correct performance of the exercise. In turn, this creates a need for a tool that can ensure users 
correctly perform the exercise without supervision. This project aims to develop a tool that can 
provide feedback and corrections for the prescribed physical rehabilitation exercise.

\subsection{Inputs and Outputs}

\textbf{Inputs:} A recording of the user performing their physical rehabilitation exercise.\\
\textbf{Outputs:} Feedback or corrections of the demonstrated movement, along with adjustments
to the form as needed. 

\subsection{Stakeholders}


  \subsubsection{Primary Stakeholders}
    \textbf{End users/Patients:} \\
      The main audience for this application will be users who have been undergoing physiotherapy treatment for their right leg, and have been given a home exercise plan which outlines \textbf(exercise 1). \\    
     
  \subsubsection{Secondary Stakeholders}
    \textbf{Physiotherapists:} \\
      The application can be used as an adjunct tool for physiotherapists, allowing them to evaluate the patient's performance and provide appropriate feedback. 
      
  \subsubsection{Tertiary Stakeholders}
    \textbf{Regulatory authorities:}
      They will be ensuring and assessing that the application is working properly and providing accurate feedback. \\
     \textbf{Other healthcare providers:}
       Specialists such as physiatrists and registered massage therapists may benefit from the information provided by the application about their patients. 

\subsection{Environment}
  \subsubsection{Software}
    The application will be built using a Python environment, with the use of OpenCV and MediaPipe. \\
  \subsubsection{Hardware}
    The application will run on an iOS device with a camera recording ability.


\section{Goals}

\section{Stretch Goals}

\section{Extras}

\wss{For CAS 741: State whether the project is a research project. This
designation, with the approval (or request) of the instructor, can be modified
over the course of the term.}

\wss{For SE Capstone: List your extras.  Potential extras include usability
testing, code walkthroughs, user documentation, formal proof, GenderMag
personas, Design Thinking, etc.  (The full list is on the course outline and in
Lecture 02.) Normally the number of extras will be two.  Approval of the extras
will be part of the discussion with the instructor for approving the project.
The extras, with the approval (or request) of the instructor, can be modified
over the course of the term.}

\newpage{}

\section*{Appendix --- Reflection}

\wss{Not required for CAS 741}

\input{../Reflection.tex}

\begin{enumerate}
    \item What went well while writing this deliverable? 
    \item What pain points did you experience during this deliverable, and how
    did you resolve them?
    \item How did you and your team adjust the scope of your goals to ensure
    they are suitable for a Capstone project (not overly ambitious but also of
    appropriate complexity for a senior design project)?
\end{enumerate}  

\bibliographystyle {plainnat}
\bibliography{../../refs/References}

\end{document}